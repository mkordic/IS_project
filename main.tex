% !TEX encoding = UTF-8 Unicode
\documentclass[a4paper]{article}

\usepackage{color}
\usepackage{url}
\usepackage[T2A]{fontenc} % enable Cyrillic fonts
\usepackage[utf8]{inputenc} % make weird characters work
\usepackage{graphicx}
\usepackage{float}
\usepackage{enumitem}

\usepackage[english,serbian]{babel}
%\usepackage[english,serbianc]{babel} %ukljuciti babel sa ovim opcijama, umesto gornjim, ukoliko se koristi cirilica

\usepackage[unicode]{hyperref}
\hypersetup{colorlinks,citecolor=green,filecolor=green,linkcolor=blue,urlcolor=blue}

\usepackage{listings}

%\newtheorem{primer}{Пример}[section] %ćirilični primer
\newtheorem{primer}{Primer}[section]

\definecolor{mygreen}{rgb}{0,0.6,0}
\definecolor{mygray}{rgb}{0.5,0.5,0.5}
\definecolor{mymauve}{rgb}{0.58,0,0.82}

\lstset{ 
  backgroundcolor=\color{white},   % choose the background color; you must add \usepackage{color} or \usepackage{xcolor}; should come as last argument
  basicstyle=\scriptsize\ttfamily,        % the size of the fonts that are used for the code
  breakatwhitespace=false,         % sets if automatic breaks should only happen at whitespace
  breaklines=true,                 % sets automatic line breaking
  captionpos=b,                    % sets the caption-position to bottom
  commentstyle=\color{mygreen},    % comment style
  deletekeywords={...},            % if you want to delete keywords from the given language
  escapeinside={\%*}{*)},          % if you want to add LaTeX within your code
  extendedchars=true,              % lets you use non-ASCII characters; for 8-bits encodings only, does not work with UTF-8
  firstnumber=1000,                % start line enumeration with line 1000
  frame=single,	                   % adds a frame around the code
  keepspaces=true,                 % keeps spaces in text, useful for keeping indentation of code (possibly needs columns=flexible)
  keywordstyle=\color{blue},       % keyword style
  language=Python,                 % the language of the code
  morekeywords={*,...},            % if you want to add more keywords to the set
  numbers=left,                    % where to put the line-numbers; possible values are (none, left, right)
  numbersep=5pt,                   % how far the line-numbers are from the code
  numberstyle=\tiny\color{mygray}, % the style that is used for the line-numbers
  rulecolor=\color{black},         % if not set, the frame-color may be changed on line-breaks within not-black text (e.g. comments (green here))
  showspaces=false,                % show spaces everywhere adding particular underscores; it overrides 'showstringspaces'
  showstringspaces=false,          % underline spaces within strings only
  showtabs=false,                  % show tabs within strings adding particular underscores
  stepnumber=2,                    % the step between two line-numbers. If it's 1, each line will be numbered
  stringstyle=\color{mymauve},     % string literal style
  tabsize=2,	                   % sets default tabsize to 2 spaces
  title=\lstname                   % show the filename of files included with \lstinputlisting; also try caption instead of title
}

\begin{document}

\title{Informacioni sistem servisa za avione\\ \small{Seminarski rad u okviru kursa\\Informacioni sistemi\\ Matematički fakultet}}

\author{Mirko Kordić, Dragana Zdravković, Mihajlo Vićentijević\\ mirko22kordic@gmail.com, dragana.zdravkovic602@gmail.com, \\mihajlovicent@gmail.com}

%\date{9.~april 2015.}

\maketitle

\tableofcontents

\newpage

\section{Uvod}
\label{sec:uvod}
U savremenoj avioindustriji, ključni faktor za uspeh svake avio kompanije leži u osiguranju bezbednih letova i održavanju redovnog avio-saobraćaja. U postizanju ovih ciljeva, efikasno i redovno održavanje aviona ima presudnu ulogu. Podaci pokazuju da su kvarovi na avionima uzrok za približno trećinu otkazanih letova ili letova sa značajnim kašnjenjima. Ovi neplanirani problemi donose avio-kompanijama velike troškove. Međutim, za razliku od faktora na koje se ne može uticati, kao što su vremenski uslovi ili štrajkovi aerodromskog osoblja, detaljnim planiranjem održavanja aviona moguće je značajno umanjiti ove troškove i povećati bezbednost putnika.

Razlikujemo dva osnovna tipa servisa: redovni i vanredni. Redovni servisi se dalje dele na tri različita podtipa:
\begin{itemize}
    \item dnevni - razmak između dva dnevna servisa je maksimalno 48h
    \item nedeljni - razmak između dva nedeljna servisa je maksimalno jedna nedelja
    \item ciklusni - ovi servisi se održavaju kada avion dođe do određenog broja ciklusa motora (jedno paljenje i gašenje motora predstavlja jedan ciklus) ili nakon određenog broja sati rada motora.
\end{itemize}

Vanredni servisi su oni servisi koji se održavaju nakon što senzori aviona očitaju neku grešku na motoru ili nekoj drugoj komponenti. Oni zahtevaju hitno reagovanje kako bi se avion što pre vratio u saobraćaj. 

U našem radu neće biti obrađeni slučajevi kada se avion pokvari na nekom drugom aerodromu, u kom nije naš servis, jer bi obuhvatanje ovog slučaja  eksponencijalno povećalo složenost sistema.

U ovom radu će biti prikazani i detaljno opisani samo osnovni (i uprošćeni) poslovni procesi za jedan servis za održavanje aviona.
    


\subsection{Korisnici sistema}
\label{subsec:korisnici_sistema}
Svi korisnici ovog sistema su zaposleni u servisu za avione. Razlikujemo tri različita korisnika:
\begin{itemize}
    \item Upravnik(Šef smene) - zaduzen je za pravljenje rasporeda servisa aviona, kao i za pravljenje rasporeda radnog vremena tehničara.
    \item Tehničar - zaduzen je za održavanje aviona, kao i za vođenje servisne istorije.
    \item Dobavljač - vodi računa o dostupnosti rezervnih delova i naručuje delove za vanredne servise.
\end{itemize}
Tehničara i dobavljača ima više u jednoj smeni, ali postoji samo jedan upravnik.

Radi jednostavnosti, pretpostavljamo da su svi tehničari podjednako stručni i ne pravimo razliku izmedju različitih podtipova (limari, električari itd...). Ova specijalizacija bi nepotrebno zakomplikovala sistem, a ne bi povećala sam kvalitet istog.

\subsection{Sistem za automatsko naručivanje delova}
\label{subsec:sistem_za_automatsko_narucivanje_delova}
Još jedan bitan deo sistema koji se ne može svrstati u korisnike jeste sistem za automatsko naručivanje delova. Glavna uloga ovog sistema je da naručuje delove za redovne servise. Sistem se pokreće svaki put kada je napravljena promena u rasporedu servisa aviona, i ako je dodat neki novi servis u odnosu na prethodni put, naručuje delove za njega. Delovi koji su potrebni za svaki servis određeni su specifikacijom određenog tipa aviona, sa čime je sistem upoznat. Osim toga, pomaže i u naručivanju delova za vanredni servis, tako što se sistem pokreće na zahtev dobavljača (vidi odeljak \ref{subsec:narucivanje_vanredan_servis}). Delovi se naručuju od firmi sa kojima avio kompanija ima potpisan ugovor. Sistem pravi narudžbenicu na osnovu fiksnog šablona (samo se menja deo koji se naručuje) definisanog u ugovoru sa firmom. Na slici \ref{fig:bpmn_automatsko_narucivanje} možete videti bolje kako funkcionise ovaj sistem.

\begin{figure}[H]
\begin{center}
\includegraphics[scale=0.55, width = 1.1\textwidth]{Dijagrami/BPMN_Dijagrami/BPMN_automatsko_narucivanje.png}
\end{center}
\caption{Sistem za automatsko naručivanje delova}
\label{fig:bpmn_automatsko_narucivanje}
\end{figure}

\subsection{Korišćeni alati}
\label{subsec:korisceni_alati}
Tokom izrade rada korišćeni su sledeći dijagrami:
    \begin{itemize}
        \item BPMN (\textit{Business Proccess Model and Notation}) dijagrami
        \item UML (\textit{Unified Modeling Language}) dijagrami:
            \begin{enumerate}
                \item Strukturni dijagram:
                    \begin{itemize}
                        \item Dijagram isporuke
                        \item Dijagram klasa podataka
                        \item Dijagram komponenti
                    \end{itemize}
                \item Dijagram ponašanja:
                    \begin{itemize}
                        \item Dijagram slučaja upotrebe
                        \item Dijagram aktivnosti
                        \item Dijagram sekvenci
                    \end{itemize}
            \end{enumerate}
        \item DFD (\textit{Data Flow Diagram}):
            \begin{enumerate}
                \item Dijagram konteksta
                \item Dijagram toka podataka nivoa 0
            \end{enumerate}
    \end{itemize}

Za izradu ovih dijagrama korišćeni su sledeći alati:
    \begin{itemize}
        \item Visual Paradigm Online Edition
        \item Visual Paradigm Community Edition
        \item WebSequenceDiagrams
    \end{itemize}

Za izradu korisničkog interfejsa korišćeni su HTML i CSS.

\section{Dijagrami toka podataka}
\label{sec:dijagrami_toka_podataka}
Na slikama \ref{fig:dtp_dijagram_konteksta} i \ref{fig:dtp_nivoa_0} možete videti dijagrame koji opisuju tok podataka našeg sistema. 

\begin{figure}[H]
\begin{center}
\includegraphics[scale=0.55]{Dijagrami/Dijagrami_toka_podataka/DTP_dijagram_konteksta.png}
\end{center}
\caption{Dijagram konteksta}
\label{fig:dtp_dijagram_konteksta}
\end{figure}

\begin{figure}[H]
\begin{center}
\includegraphics[scale=0.7, width = 1.1\textwidth]{Dijagrami/Dijagrami_toka_podataka/DTP_nivo_0.png}
\end{center}
\caption{Dijagram toka podataka nivoa 0}
\label{fig:dtp_nivoa_0}
\end{figure}

\section{Slučajevi upotrebe}
\label{sec:sluvajevi_upotrebe}
U ovom poglavlju ćemo detaljno opisati osnovne slučajeve upotrebe našeg sistema, pružajući uvid u ključne scenarije i funkcionalnosti koje naš sistem omogućava. Radi bolje preglednosti, nakon slučaja upotrebe celog sistema (slika \ref{fig:ceo_sistem}), ostali slučajevi upotrebe će biti grupisani na osnovu korisnika navedenih u glavi \ref{subsec:korisnici_sistema}.

\begin{figure}[H]
\begin{center}
\includegraphics[scale=0.55]{Dijagrami/Dijagrami_slucajeva_upotrebe/Dijagram_slučajeva_upotrebe_celog_informacionog_sistema.png}
\end{center}
\caption{Dijagram slučajeva upotrebe celog informacionog sistema}
\label{fig:ceo_sistem}
\end{figure}

Radi boljeg razumevanja funkcionisanja celog sistema, ispod (Slika \ref{fig:bpmn_dijargram_saradnje}) je priložen dijagram saradnje izmedju upravnika i dobavljača.

\begin{figure}[H]
\begin{center}
\includegraphics[scale=0.5]{Dijagrami/BPMN_Dijagrami/Dijagram_saradnje_upravnik_dobavljac.png}
\end{center}
\caption{Dijagram saradnje upravnika i dobavljača}
\label{fig:bpmn_dijargram_saradnje}
\end{figure}

\subsection{Prijavljivanje na sistem}
\label{subsec:prijavljivanje}

\begin{itemize}
    \item Kratak opis: Iako ovo nije klasičan slučaj upotrebe, važno je dodatno opisati proceduru samog prijavljivanja na sistem. Ovaj proces se ponavlja svaki put kada se korisnik odjavi sa aplikacije ili se računar ugasi.
    \item Akteri: Svi korisnici sistema.
    \item Preduslovi: Aplikacija je funkcionalna, a računar je povezan na internet.
    \item Postuslovi: Korisnik se prijavio na sistem i prikazan mu je odgovarajući interfejs.
    \item Osnovni tok:
        \begin{enumerate}
            \item Korisnik unosi svoje korisničko ime (koje predstavlja njegov ID) i svoju šifru.
            \item Ako je šifra ispravna, korisniku se prikazuje odgovarajući interfejs.
        \end{enumerate}
    \item Alternativni tokovi:
     \begin{enumerate}
            \item  U koliko je lozinka nije dobra, dobija obavestenje o tome i trazi mu se ponovan unos lozinke.
    \end{enumerate}
    \item Specijalni uslovi: /
\end{itemize}

\begin{figure}[H]
\begin{center}
\includegraphics[scale=0.55, width = 1.1\textwidth]{Dijagrami/Dijagrami_sekvence/Dijagram_sekvence_prijavljivanje_na_sistem.png}
\end{center}
\caption{Prijavljivanje na sistem}
\label{fig:prijavljivanje}
\end{figure}

\subsection{Slučaj upotrebe: Raspored tehničara}
\label{subsec:raspored_tehnicara}
\begin{itemize}
    \item \textbf{Kratak opis}: Sistem omogućava upravi da vidi dostupne tehničare, na osnovu čega se pravi raspored njihovog radnog vremena. Tehničari se raspoređuju u hangare. U svakom hangaru mora da bude raspoređeno minimalno 5 tehničara, a maksimalno 10. Smene traju 12 sati.
    \item \textbf{Akteri}: Upravnik
    \item \textbf{Preduslovi}: Sistem je aktivan i korisnik je prijavljen kao upravnik.
    \item \textbf{Postuslovi}: Raspored je napravljen i vidljiv svim tehničarima.
    \item \textbf{Osnovni tok}:
        \begin{enumerate}
            \item Klikom na sekciju $"$Raspored$"$ otvara se kalendar sa radnim danima u tom mesecu.
            \item Odabirom radnog dana otvara se prozor gde je potrebno uneti ID, ime i prezime, kao i radno vreme tehničara, i hangar u koji je tehničar raspoređen.
            \item Klikom na polje ID otvara mu se padajući meni sa svim podacima o dostupnim tehničarima u sistemu (ime, prezime, broj dodeljenih radnih sati, maksimalan broj radnih sati, minimalan broj radnih sati).
            \item Odabirom ID tehničara, sva polja iz 2. tačke se automatski popunjavaju, osim radnog vremena i hangara koje je potrebno eksplicitno navesti. Prvo se unosi radno vreme, a nakon toga se automatski nude hangari na kojim nema raspoređen dovoljan broj tehničara.
            \item Klikom na $"$Dodaj tehničara$"$, uneti tehničar se raspoređuje za izabrani dan. Broj radnih sati se automatski ažurira za izabranog tehničara.
        \end{enumerate}
    \item \textbf{Alternativni tokovi}:
        \begin{enumerate}
            A1) U slučaju da neki tehničar iznenada nije dostupan za neki period, potrebno je izmeniti raspored.
        \end{enumerate}
    \item \textbf{Specijalni uslovi}: Tehničar se ne može rasporediti ukoliko nije dostupan. Tehničar nije dostupan ukoliko mu je ispunjen maksimalan broj radnih sati za taj mesec ili ako je na odmoru/bolovanju. Tehničari vide raspored do poslednjeg popunjenog dana.
\end{itemize}

Slike \ref{fig:ds_raspored_tehnicara} i \ref{fig:da_raspored_tehnicara} predstavljaju prateće dijagrame za ovaj slučaj upotrebe.
\begin{figure}[H]
\begin{center}
\includegraphics[scale=0.5, width = 1.1\textwidth]{Dijagrami/Dijagrami_sekvence/Dijagram_sekvence_rasporeda_tehničara.png}
\end{center}
\caption{Dijagram sekvence rasporeda tehničara}
\label{fig:ds_raspored_tehnicara}
\end{figure}

\begin{figure}[H]
\begin{center}
\includegraphics[scale=0.5]{Dijagrami/Dijagrami_aktivnosti/Dijagram_aktivnosti_raspored_tehničara.png}
\end{center}
\caption{Dijagram aktivnosti rasporeda tehničara}
\label{fig:da_raspored_tehnicara}
\end{figure}

\begin{figure}[H]
\begin{center}
\includegraphics[scale=0.8, width = 1.3\textwidth]{Dijagrami/BPMN_Dijagrami/BPMN_rasporeda_tehničara.png}
\end{center}
\caption{BPMN dijagram - rasporeda tehničara}
\label{fig:da_raspored_tehnicara}
\end{figure}

\subsection{Slučaj upotrebe: Planiranje održavanja aviona - \\redovni servis}
\label{subsec:redovan_servis}
\begin{itemize}
    \item \textbf{Kratak opis}: Sistem omogućava upravi da pravi planove redovnih servisa aviona.
    \item \textbf{Akteri}: Upravnik
    \item \textbf{Preduslovi}: Sistem je aktivan, a korisnik je član uprave. Korisnik ima informacije o svim avionima.
    \item \textbf{Postuslovi}: Plan je napravljen. Tehničari vide plan održavanja. Dobavljači vide plan održavanja.
    \item \textbf{Osnovni tok}: 
    \begin{enumerate}
        \item Klikom na sekciju $"$Planiranje servisa$"$ otvara se meni sa svim avionima.

        \item Klikom na neki od aviona otvaraju se informacije o tom avionu kao što su proizvođač, model, godina proizvodnje, broj ciklusa motora od poslednjeg servisa, broj sati rada motora od poslednjeg servisa, datum poslednjeg dnevnog servisa, nedeljnog servisa i velikog servisa...

        \item Klikom na dugme $"$Zakaži servis$"$ mu se prikazuju tri opcije: $"$Dnevni servis$"$, $"$Nedeljni servis$"$ i $"$Veliki servis$"$ (na osnovu broja ciklusa motora ili radnih sati)
        \begin{enumerate}[label=3.1.\arabic*]
            \item Ukoliko je odabran $"$Dnevni servis$"$, on mora da bude zakazan u narednih 48 sati. Klikom na polje $"$Vreme servisa$"$ bira neki od slobodnih termina (avion nije na letu) u narednih 48 sati.
            \item Nakon što je odabrao termin (svaki dnevni servis traje 3h), dobija listu slobodnih hangara, klikom na polje $"$Hangar$"$, u izabrano vreme i opredeljuje se za jedan (na avionu rade tehničari koji su u to vreme raspoređeni u izabranom hangaru). Hangar se automatski rezerviše za odabrani period.
        \end{enumerate}
        \begin{enumerate}[label=3.2.\arabic*]
            \item Ukoliko je odabran $"$Nedeljni servis$"$, on može biti zakazan u narednih sedam dana. Klikom na polje $"$Vreme servisa$"$ bira neki od slobodnih termina (avion nije na letu) u narednih nedelju dana.
            \item Nakon što je odabrao termin (svaki nedeljni servis traje 6h), dobija listu slobodnih hangara, klikom na polje $"$Hangar$"$, u izabrano vreme i opredeljuje se za jedan (na avionu rade tehničari koji su u to vreme raspoređeni u izabranom hangaru). Hangar se automatski rezerviše za odabrani period.
        \end{enumerate}

         \begin{enumerate}[label=3.3.\arabic*]
            \item Klik na polje "Veliki servis" je moguć samo ukoliko je neki od parametara broj ciklusa motora ili broj sati rada motora blizu granice za servisiranje. Korisniku je automatski preporučen datum kada je predviđeno da se servis odradi. Trajanje servisa nije vremenski ograničeno.
            \item Klik na polje $"$Veliki servis$"$ je moguć samo ukoliko je neki od parametara broj ciklusa motora ili broj sati rada motora blizu granice za servisiranje. Korisniku je automatski preporučen datum kada je predviđeno da se servis odradi. Trajanje servisa nije vremenski ograničeno.
        \end{enumerate}
        \item Klikom na dugme $"$Potvrdi$"$, korisnik zakazuje servis.
        
    \end{enumerate}
    \item \textbf{Alternativni tokovi}:
        \begin{itemize}
        A1) Ukoliko je $"$hijerarhijski slabiji$"$ servis zakazan blizu $"$jačeg$"$ servisa, korisniku se nudi opcija da ga promoviše u jači. \\
        A2) Ukoliko odbije tu opciju, dobija poruku $"$Neuspešno zakazivanje servisa$"$.
        \end{itemize}
    \item \textbf{Specijalni uslovi}: Prilikom obavljanja $"$hijerarhijski jačeg$"$ servisa ažurira se i vreme servisa za sve slabije tipove servisa.
\end{itemize}

\begin{figure}[H]
\begin{center}
\includegraphics[scale=0.55, width = 1.1\textwidth]{Dijagrami/Dijagrami_sekvence/Dijagram_sekvence_redovnog_servisa.png}
\end{center}
\caption{Slučaj upotrebe: Planiranje održavanja aviona - redovan servis}
\label{fig:ds_redovan_servis}
\end{figure}

\subsection{Slučaj upotrebe: Planiranje održavanja aviona - \\vanredni servis}
\label{subsec:vanredni_servis}
\begin{itemize}
    \item \textbf{Kratak opis}: Sistem omogućava upravi da rukovodi vanrednim servisima aviona.
    \item \textbf{Akteri}: Upravnik
    \item \textbf{Preduslovi}: Sistem je u aktivnom stanju. Avion opremljen ispravnim senzorima koji funkcionišu. Ukoliko senzor detektuje grešku, sistem aviona je sposoban da prepozna tip greške i prenese informaciju ovom sistemu.
    \item \textbf{Postuslovi}: Vanredni servis je zakazan. Tehničari vide plan održavanja. Dobavljači vide plan održavanja.
    \item \textbf{Osnovni tok}:
    \begin{enumerate}
        \item Na ekranu je, u vidu prozora, prikazano obaveštenje o kvaru aviona. Klikom na dugme $"$Prikaži više$"$ otvaraju se detaljnije informacije o kvaru.
        \item Uprava klikom na dugme $"$Pošalji upit o rezervnim delovima$"$ šalje ove informacije o kvaru dobavljačima.
        \item Kada dobije odgovor o dostupnosti delova (od dobavljača), prikazuje mu se poruka u prozoru. Klikom na dugme $"$Zakaži servis$"$ koje se nalazi ispod poruke mu se otvara dijalog za zakazivanje servisa pokvarenom avionu.
        \item Polje vreme servisa je automatski popunjeno na prvi slobodan termin (u obzir se uzima i dostupnost potrebnih delova). Klikom na njega može da se i odabere neko drugo vreme.
        \item Kada se odabere vreme, klikom na polje $"$Hangar$"$ bira se neki od slobodnih hangara za odabrani termin.
        \item Klikom na dugme $"$Potvrdi$"$, korisnik zakazuje servis. Servis je dodat u plan održavanja.
    \end{enumerate}
    
    \item \textbf{Alternativni tokovi}:
        \begin{enumerate}
            A1) Ukoliko je vanredni servis zakazan $"$blizu$"$ dnevnog ili nedeljnog servisa ($\pm$1 dan), korisniku se nudi opcija da otkaže redovan servis.
        \end{enumerate}
    \item \textbf{Specijalni uslovi}: Prilikom obavljanja vanrednog servisa, ažuriraju se vremena za dnevni i nedeljni servis u bazi podataka.
\end{itemize}

Komunikacija koja se odvija sa dobavljačima je detaljno prikazana na slici \ref{fig:bpmn_dijargram_saradnje}.

\begin{figure}[H]
\begin{center}
\includegraphics[scale=0.6, width = 1.0\textwidth]{Dijagrami/Dijagrami_sekvence/Dijagram_sekvence_zakazivanje_vanrednog_servisa.png}
\end{center}
\caption{Dijagram sekvence zakazivanja vanrednog servisa}
\label{fig:ds_zakazivanje_vanrednog_servisa}
\end{figure}


\begin{figure}[H]
\begin{center}
\includegraphics[scale=0.6, width = 1.0\textwidth]{Dijagrami/Dijagrami_aktivnosti/Dijagram_aktivnosti_vanrednog_servisa.png}
\end{center}
\caption{Dijagram aktivnosti zakazivanja vanrednog servisa}
\label{fig:da_zakazivanje_vanrednog_servisa}
\end{figure}

\subsection{Slučaj upotrebe - servisiranje aviona}
\label{subsec:servisiranje_aviona}
\begin{itemize}
    \item \textbf{Kratak opis}: Sistem omogućava tehničarima da vide raspored servisa i informacije o servisu. Tehničar servisira onaj avion čiji je servis zakazan za vreme njegovog radnog vremena u hangaru u kom je on raspoređen.
    \item \textbf{Akteri}: Tehničar
    \item \textbf{Preduslovi}: Sistem je aktivan. Raspored servisa je napravljen i dostupan tehničarima.
    \item \textbf{Postuslovi}: Tehničar je uspešno obavio servis aviona i uneo servis u evidenciju. Delovi korišćeni na servisu se brišu iz baze.
    \item \textbf{Osnovni tok}:
        \begin{enumerate}
            \item Tehničar klikom na sekciju $"$Raspored servisa$"$ vidi sve zakazane servise. Klikom na određen servis vidi dodatne informacije za taj servis (tip aviona, proizvođač, godina, uputstva od proizvođača).
            \item Kada dođe vreme za određen servis, sistem automatski postavlja status servisa na $"$U toku$"$.
            \item Kada je servis završen, klikom na dugme $"$Završi servis$"$ otvara se prozor u kome treba da unese informacije o servisu koje će biti sačuvane kao servisna istorija aviona.
            \item U polju $"$Urađeno na servisu$"$ navodi sve što je urađeno u tom servisu. Polja kao što su tip servisa, datum i vreme servisa su unapred popunjena.
            \item Klikom na dugme $"$Zapamti servis$"$, servis se čuva u servisnoj istoriji i status mu je promenjen u $"$Završen$"$.
            \item Nakon toga, unosi šifre delova koje je koristio u polje $"$Korišćeni delovi$"$, jednu ispod druge, pri čemu mu $"$autocomplete$"$ nudi opcije.
        \end{enumerate}
    \item \textbf{Alternativni tokovi:}
        \begin{enumerate}
            A1) Ako je polje $"$Urađeno na servisu$"$ ili $"$Korišćeni delovi$"$ prazno, a klikne se na dugme $"$Zapamti servis$"$, dobija se poruka o grešci.\\
            A2) Neki od unetih delova nisu pronađeni u bazi podataka, što rezultuje porukom o grešci.
        \end{enumerate}
    \item \textbf{Specijalni uslovi}: /
\end{itemize}

\begin{figure}[H]
\begin{center}
\includegraphics[scale=0.6, width = 1.0\textwidth]{Dijagrami/Dijagrami_sekvence/Dijagram_sekvence_servisiranje_aviona.png}
\end{center}
\caption{Dijagram sekvence servisiranja aviona}
\label{fig:ds_servisiranje_aviona}
\end{figure}

\subsection{Slučaj upotrebe - upravljanje zalihama}
\label{subsec:redovan_servis}
\begin{itemize}
    \item \textbf{Kratak opis}: Ovaj slučaj upotrebe fokusira se na upravljanje zalihama delova za avion od strane dobavljača, kako bi se osigurala dostupnost potrebnih komponenti za redovna servisiranja aviona. Sistem za automatsko naručivanje delova, osmišljen je za praćenje, upravljanje i obezbeđivanje delova potrebnih za redovne servise.
    \item \textbf{Akteri}: Dobavljač
    \item \textbf{Preduslovi}: Postojanje sistema za automatsko upravljanje delovima. Dobavljač vidi raspored redovnih servisa i informacije o trenutnom stanju zaliha. Sistem je aktivan.
    \item \textbf{Postuslovi}: Ažurirane informacije o zalihama za sve redovne servise. 
    \item \textbf{Osnovni tok}:
        \begin{enumerate}
            \item Klikom na sekciju $"$Narudžbine$"$ prikazuju mu se sve narudžbenice koje je sistem za automatsko upravljanje napravio.
            \item Klikom na narudžbenicu prikazuju mu se sve informacije o njoj (status narudžbenice, servis za koji je napravljena, vreme stizanja, informacije o naručenom delu itd...).
            \item Ukoliko je status narudžbenice $"$Isporučena$"$, dostupno mu je dugme $"$Pridruži servisu$"$. Klikom na dugme, delovi se automatski pridružuju servisu za koji su namenjeni i stanje delova se ažurira (zapisuje se u bazi podataka da je deo pridruzen servisu).
        \end{enumerate}
    \item \textbf{Alternativni tokovi}: /
    \item \textbf{Specijalni uslovi}: Sistem mora pružati tačne i ažurirane informacije o dostupnosti delova. Informacije o narudžbenicama moraju biti transparentne, uključujući procenjeni datum dolaska delova.
\end{itemize}

\begin{figure}[H]
\begin{center}
\includegraphics[scale=0.5]{Dijagrami/BPMN_Dijagrami/BPMN_dijagram_procesa_upravljanje_narudzbenicama.png}
\end{center}
\caption{Dijagram procesa - Upravljanje zalihama}
\label{fig:BPMN_upravljanje_zalihama}
\end{figure}

\subsection{Slučaj upotrebe - naručivanje delova za vanredni servis}
\label{subsec:narucivanje_vanredan_servis}
\begin{itemize}
    \item \textbf{Kratak opis}: Ovaj slučaj upotrebe pokriva naručivanje delova kada dođe do kvara aviona.
    \item \textbf{Akteri}: Dobavljač
    \item \textbf{Preduslovi}: Sistem funkcioniše. Uprava je dobila obaveštenje o kvaru aviona (sa informacijama o tome šta je pokvareno) i poslala upit dobavljaču o dostupnosti delova.
    \item \textbf{Postuslovi}: Uprava je obaveštena o dostupnosti delova (da li je deo dostupan ili očekivano vreme isporuke).
    \item \textbf{Osnovni tok}: 
        \begin{enumerate}
            \item Stiže obaveštenje od uprave da je došlo do kvara aviona. Klikom na obaveštenje dobijaju se informacije o pokvarenim delovima.
            \item Klikom na dugme $"$Proveri dostupnost$"$, dobavljač dobija informaciju o dostupnosti datih delova.
            \item Ukoliko su svi delovi dostupni, omogućeno mu je da odmah pošalje odgovor upravi da su svi potrebni delovi dostupni, klikom na dugme $"$Pošalji odgovor$"$.
            \item Ukoliko neki deo nije dostupan, pored njega postoji dugme $"$Naruči deo$"$. Klikom na ovo dugme sistem za automatsko naručivanje se pokreće i pravi narudžbenicu. Kada sistem naruči deo, dobija potvrdu.
            \item Kada su svi delovi ili dostupni ili naručeni, moguće je da klikne na dugme $"$Pošalji odgovor$"$, čime šalje odgovor upravi o dostupnosti delova (odgovor sadrži informacije o dostupnosti delova; da li je deo dostupan ili procenjeno vreme dolaska).
        \end{enumerate}
    \item \textbf{Alternativni tokovi}: /
    \item \textbf{Specijalni uslovi}: Sistem mora pružati tačne i ažurirane informacije o dostupnosti delova. Informacije o narudžbenicama moraju biti transparentne, uključujući procenjeni datum dolaska delova.
\end{itemize}

Komunikacija koja se odvija sa upravnikom je detaljno prikazana na slici \ref{fig:bpmn_dijargram_saradnje}.
\begin{figure}[H]
\begin{center}
\includegraphics[scale=0.4]{Dijagrami/Dijagrami_sekvence/Dijagram_sekvenci_narucivanje_delova.png}
\end{center}
\caption{Dijagram sekvence naručivanja delova}
\label{fig:ds_narucivanje_delova}
\end{figure}

\pagebreak

\section{Baza podataka}
\label{sec:baza_podataka}

Na osnovu analize slučajeva upotrebe, prepoznali smo sledeće entitete o kojima je neophodno čuvati informacije u bazi podataka:
\begin{itemize}
    \item Zaposleni
        \begin{itemize}[label=-]
            \item Upravnici
            \item Tehničari
            \item Dobavljači
        \end{itemize}
    \item Avion
    \item Hangar
    \item Narudžbenica
    \item Servis
    \item Raspored
    \item Deo
\end{itemize}

Na slici ispod možete videti detaljniji opis baze podataka, predstavljen uz pomoć dijagrama klasa podataka.

\begin{figure}[H]
\begin{center}
\includegraphics[scale=0.6, width = 1.2\textwidth]{Dijagrami/Dijagram_klasa_podataka/Dijagram_klasa_podataka.png}
\end{center}
\caption{Dijagram klase podataka}
\label{fig:dks}
\end{figure}

Provera oko fonda radnih časova za tehničara je odvojena od baze podataka (u smislu da ne postoji trigger ili procedura koja to radi, već se to reguliše na nivou koda aplikacije). Uz pomoć $schedulera$ SUBP-a, svakog prvog u mesecu se za sve radnike kolona broj radnih sati u mesecu postavlja na nulu.

\section{Arhitektura informacionog sistema}
\label{sec:arhitektura}
Pažljivo razmatrajući funkcionalne, nefunkcionalne i bezbednoste zahteve informacionog sistema, odlučili smo se za slojevitu klijent-server arhitekturu. Lakši razvoj i održavanje, kao i povećana bezbednost (mogućnost implementiranja različitih bezbednosnih mehanizama na svakom od slojeva) su jedni od glavnih razloga zašto smo se opredelili baš za slojevitu arhitekturu. Razlikujemo sledeća tri sloja:
\begin{enumerate}
    \item Prezentacioni sloj - Najviši sloj, omogućava prikaz korisničkog interfejsa i njegovu odzivnost. Odgovoran je za unos i prikaz podataka. Korisnički interfejs je različit za različite korisnike, što rezultira različitim funkcionalnostima. Tehnologije koje se koriste na ovom sloju su: Java, JavaFX.
    \item Logički sloj - Srednji sloj, u njemu se nalazi sva logika aplikacije. Služi kao posrednik između prezentacionog i sloja podataka. $"$Preuređuje$"$ podatke koje dobija od jedne strane, tako da su pogodni za prikaz, obradu ili čuvanje na drugoj strani. Tehnologija koja se koristi na ovom sloju je Java.
    \item Sloj podataka - Najvisi sloj, na kome se nalazi relacioni SUBP i relaciona baza. Konkretan SUBP na ovom sloju je MySQL.
\end{enumerate}

Aplikacija je tipa desktop aplikacije i biće raspoređena na tri računara: jedan klijentski i dva serverska računara. Jedan od serverskih računara će biti aplikativni server, dok će drugi služiti kao server baze podataka. Za fizičko razdvajanje aplikativnog servera i servera baze podataka smo se odlučili pre svega zbog boljih performansi. Odvajanje baze podataka od aplikativnog servera omogućava svakom serveru da koristi sve resurse samo za posao koji obavlja. Aplikacija koja se instalira na klijentskom računaru ima samo korisnički interfejs i deo logike koji omogućava korišćenje istog, dok će veći deo logičkog sloja biti na aplikativnom serveru. Ovu odluku opravdavamo lakšim održavanjem, jer prilikom promene u logici aplikacije nije uvek neophodno ažurirati svakog klijenta, kao i povećanom bezbednošću. Centralizovanje logike na backend serveru omogućava bolju kontrolu nad podacima i pristupom. Na serveru baze podataka koristićemo MySQL SUBP zbog njegovih performansi i jer je open-source softver koji je besplatan, i ima veliku zajednicu korisnika.

Na slikama \ref{fig:dijagram_komponenti_sistema} i \ref{fig:isporuka} mozete videti dijagram komponenti sistema i dijagram isporuke. 

\begin{figure}[H]
\begin{center}
\includegraphics[scale=0.6, width = 1.2\textwidth]{Dijagrami/Dijagrami_komponenti_sistema/Dijagram_komponenti_sistema.png}
\end{center}
\caption{Dijagram komponenti sistema}
\label{fig:dijagram_komponenti_sistema}
\end{figure}

\begin{figure}[H]
\begin{center}
\includegraphics[scale=0.6, width = 1\textwidth]{Dijagrami/Dijagrami_isporuke/Dijagram_isporuke.png}
\end{center}
\caption{Dijagram isporuke}
\label{fig:isporuka}
\end{figure}

\section{Korisnički interfejs}
U ovom poglavlju će biti prikazan prototip korisničkog interfejsa.

\begin{figure}[H]
\begin{center}
\includegraphics[scale=0.6]{UI/prijava.png}
\end{center}
\caption{Prijava na sistem}
\label{fig:ui_prijava}
\end{figure}

\begin{figure}[H]
\begin{center}
\includegraphics[scale=0.6, width = 1\textwidth]{UI/zakazivanje_tehnicara_1.png}
\end{center}
\caption{Pravljanje rasporeda tehničara}
\label{fig:ui_raspored_tehnicara_1}
\end{figure}

\begin{figure}[H]
\begin{center}
\includegraphics[scale=0.9, width = 1.2\textwidth]{UI/zakazivanje tehnicara_2.png}
\end{center}
\caption{Pravljanje rasporeda tehničara}
\label{fig:ui_raspored_tehnicara_2}
\end{figure}

\begin{figure}[H]
\begin{center}
\includegraphics[scale=0.3]{UI/planiranje_servisa.png}
\end{center}
\caption{Pravljanje rasporeda redovnih servisa}
\label{fig:ui_raspored_redovnih_servisa}
\end{figure}

\begin{figure}[H]
\begin{center}
\includegraphics[scale=0.3]{UI/vandredni_servis_1.png}
\end{center}
\caption{Vanredni servis}
\label{fig:ui_vanredni_1}
\end{figure}

\begin{figure}[H]
\begin{center}
\includegraphics[scale=0.3]{UI/vandredni_servis_2.png}
\end{center}
\caption{Vanredni servis}
\label{fig:ui_vanredni_2}
\end{figure}

\begin{figure}[H]
\begin{center}
\includegraphics[scale=0.3]{UI/vandredni_servis_3.png}
\end{center}
\caption{Vanredni servis}
\label{fig:ui_vanredni_3}
\end{figure}

\begin{figure}[H]
\begin{center}
\includegraphics[scale=0.3]{UI/vandredni_servis_4.png}
\end{center}
\caption{Vanredni servis}
\label{fig:ui_vanredni_4}
\end{figure}

\begin{figure}[H]
\begin{center}
\includegraphics[scale=0.3]{UI/servisiranje_aviona.png}
\end{center}
\caption{Servisiranje aviona}
\label{fig:ui_servisiranje_aviona_1}
\end{figure}

\begin{figure}[H]
\begin{center}
\includegraphics[scale=0.3]{UI/servisiranje_aviona_2.png}
\end{center}
\caption{Servisiranje aviona}
\label{fig:ui_servisiranje_aviona_2}
\end{figure}

\begin{figure}[H]
\begin{center}
\includegraphics[scale=0.3]{UI/narudzbine.png}
\end{center}
\caption{Upravljanje zalihama}
\label{fig:ui_upravljanje_zalihama}
\end{figure}

\begin{figure}[H]
\begin{center}
\includegraphics[scale=0.3]{UI/narucivanje_delova_za_vanredni_servis_1.png}
\end{center}
\caption{Naručivanje delova za vanredni servis}
\label{fig:ui_naručivanje_1}
\end{figure}

\begin{figure}[H]
\begin{center}
\includegraphics[scale=0.35]{UI/narucivanje_delova_za_vanredni_servis_2.png}
\end{center}
\caption{Naručivanje delova za vanredni servis}
\label{fig:ui_naručivanje_2}
\end{figure}

\begin{figure}[H]
\begin{center}
\includegraphics[scale=0.3]{UI/narucivanje_delova_za_vanredni_servis_3.png}
\end{center}
\caption{Naručivanje delova za vanredni servis}
\label{fig:ui_naručivanje_3}
\end{figure}

\pagebreak

\section{Zaključak}
\label{sec:zakljucak}

U ovom projektu je razvijen predlog informacionog sistema za avio-servise unutar avio-kompanija. Kompanije trpe značajne gubitke zbog neefikasnih i nedovoljno optimizovanih procesa održavanja aviona. Naš sistem ne samo da može potencijalno smanjiti gubitke kompanije, već i unaprediti bezbednost putnika kroz redovno i detaljno servisiranje aviona.

\subsection{Dalji razvoj sistema}
\label{subsec:dalji_razvoj}

Zbog ograničenog vremena i broja članova tima, kao i složenosti problema koji rešavamo, ovaj projekat obuhvata samo ključne odnosno centralne slučajeve upotrebe vezane za avio-servise. U budućnosti je planirano raditi na sledećim funkcionalnostima:
\begin{itemize}
    \item Rešavanje problema kada se avion pokvari na aerodromu u kom nema servisa njegove avio-kompanije. Ova izmena eksponencijalno povećava složenost samog sistema.
    \item Mogućnost dobijanja različitih vrsta statistika vezanih za servisiranje aviona. Ove statistike bi bile korisne za otkrivanje stvari kao što su najčešćih uzroci kvara na avionima (što za odredjen tip aviona, što generalno).
\end{itemize}

\end{document}